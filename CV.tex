\documentclass[11pt,a4paper]{article}
\usepackage[utf8]{inputenc}
\usepackage[top=1cm, left=1.5cm, right=1.5cm, bottom=1cm]{geometry}
\usepackage{xcolor}
\usepackage{marvosym}
\usepackage{multirow}
\usepackage[hidelinks]{hyperref}
\usepackage{textcomp}
\usepackage{array}

\hypersetup{
    colorlinks=true,
    linkcolor=blue,
    filecolor=magenta,
    urlcolor=cyan,
    pdftitle={Smit_Chaudhary_CV},
    pdfpagemode=FullScreen,
    }

\urlstyle{same}

\definecolor{myBlue}{rgb}{0.23,0.45,0.8}

\title{\Huge{\textbf{\textcolor{myBlue}{Smit Chaudhary}} {\textcolor{gray}{ $\mid$ CV}} }}
\author{}
\date{}

\begin{document}
\vspace{-10cm}
\maketitle
\pagenumbering{gobble}
\begin{center}
\vspace{-2cm}
%
\Large{\textsc{\textcolor{gray}{Masters student}, Applied Physics, TU Delft}}\\
\normalsize%
\vspace{0.1cm}
\Mobilefone \phantom{i} +31 0613782478%
\phantom{x} $\bullet$ \phantom{x}
\Letter \phantom{i} \href{mailto:S.N.Chaudhary@stydent.tudelft.nl}{S.N.Chaudhary@student.tudelft.nl}%
\phantom{x} $\bullet$ \phantom{x} %
\Mundus \phantom{i} \href{http://smitchaudhary.github.io/home}{smitchaudhary.github.io}
\end{center}%
%
\vspace{0.3cm}
\color{myBlue}
\Large\textbf{\textit{Education}} \vspace{0.15cm} \normalsize
\color{gray} \hrule
\color{black}
\begin{table}[h!]
\centering
\begin{tabular}{@{\phantom{MM}}c@{\phantom{MMM}}c@{\phantom{MMMM}}c@{\phantom{}}}
\textbf{Year} & \textbf{Degree} & \textbf{Institution} \\ % & \textbf{CGPA/\%} \\
\hline
2022$^*$ & Masters of Science in Applied Physics & Delft University of Technology, Delft \\
2020 &
Bachelor of Science in Physics &
Indian Institute of Technology, Kanpur \\ % &
%7.8/10 \\
2016 & Class XII (CBSE) & New Tulip International School, Ahmedabad \\ %& 94.6\% \\
2014 & Class X (CBSE) & Kendriya Vidyalaya, Sabarmati, Ahmedabad \\ % & 10.0/10.0
\end{tabular}
\end{table}
%\vspace{0.2cm}
%\color{myBlue}
%\noindent \Large\textbf{\textit{Interests}} \vspace{0.15cm} \normalsize
%\color{gray} \hrule
%\color{black}
%\begin{center}
% \textbf{\textcolor{gray}{Near Term Quantum Algorithms}} \phantom{i} $\bullet$  %\textbf{\textcolor{gray}{Quantum Optimization}} \phantom{i} $\bullet$ \textbf{\textcolor{gray}{Quantum %Information Theory}}
%\end{center}
\color{myBlue}
\vspace{0.2cm}
\noindent \Large\textbf{\textit{Key Projects and Experience}} \vspace{0.15cm} \normalsize
\color{gray} \hrule
\color{black}
\begin{itemize}
    \item \textbf{\textit{Qubit Mapping with Quantum Enhanced Algorithm (In Progress)}} \hfill \textit{Aug'21 - Present}\\
    \texttt{MSc. Thesis:} \textit{Supervisor : Prof. Sebastian Feld, QuTech, TU Delft}
    \begin{itemize}
        \item Inspected quantum walk algorithms and its application in speeding up backtracking problem
        \item Designed a backtracking based mapping strategy and implemented a quantum walk algorithm to give a quadratic speed up over classical algorithm
        \item Currently solving routing problem using backtracking techniques and examining ways to extend quantum walk on it for beyond graphs with unbounded degree
    \end{itemize}
    \item \textbf{Barren Plateaus in QNN training with correlated Noise (In Progress)} \hfill \textit{Aug'21 - Present} \\
    \texttt{Honors Track Project:} \textit{Supervisor : Prof. Jordi Tura, Leiden University}
    \begin{itemize}
        \item Studied Barren Plateaus in QNN training landscape due to random parameter initialisation as well as due to noise. Reproduced the results for Haar random circuits and local pauli noise
        \item Examined realistic noises in quantum chips and implemented channels with correlated noise
        \item Assessed the effect of correlated noises and noise strength on barren plateau and inspecting to get a tighter upper bound under certain noise strengths for correlated noise
    \end{itemize}

    \item \textbf{\textit{Quantum Generative Adversarial Network}} \hfill \textit{April - May'21}\\
    \texttt{Course:} \textit{Applied Quantum Algorithms, Leiden University} \hfill \href{https://github.com/smitchaudhary/QGANs}{\texttt{[CODE]}}\href{https://github.com/smitchaudhary/QGANs/blob/main/Report.pdf}{\texttt{[REPORT]}}
    \begin{itemize}
        \item Reviewed Generative Adversarial Networks (GANs) and designed a quantum version of the same
        \item Extended classical Generator-Discriminator pair to one able of handling Quantum data (quantum states) and produce the desired quantum state
        \item Performed hyper-parameter optimization and exhibited the dependence of the QCBM on it
        \item Benchmarked the performance of the QGAN against classical GAN for quantum states
    \end{itemize}
    \item \textbf{\textit{Quantum Approximate Optimization Algorithms}} \hfill \textit{Nov'20-Jan'21}\\
    \texttt{Mentor:} \textit{Prof. Leonardo DiCarlo, TU Delft} \hfill
    \href{https://github.com/smitchaudhary/QAOA-MaxCut}{\texttt{[CODE]}}\href{https://github.com/smitchaudhary/QAOA-MaxCut/blob/master/Report.pdf}{\texttt{[REPORT]}}\href{https://github.com/smitchaudhary/QAOA-MaxCut/blob/master/Presentation.pdf}{\texttt{[SLIDES]}}
    \begin{itemize}
        \item Studied QAOA and its applications for a number of combinatorial optimisation problems
        \item Examined noises and built a noise model to implement QAOA for Max-Cut using simulator to determine the effect of different kinds of noises
        \item Modified the algorithm to run it on different superconducting qubits based quantum hardware (IBM's Vigo and QuTech's Starmon5) with reduced calls to the hardware
        \item Analysed the performance of the algorithm on near term machines and studied the performance with varying circuit depth and different noise models
    \end{itemize}
    \item \textbf{\textit{Entanglement distillation on noisy quantum channels}} \hfill \textit{Dec'20-Jan'21}\\
    \texttt{Mentor:} \textit{Prof. Stephanie Wehner, TU Delft} \hfill
    \begin{itemize}
        \item Investigated and compared 3 different 2-to-1 entanglement distillation protocols (EPL, DEJMPS, BBPSSW) and a 3-to-1 protocol under ideal conditions
        \item Implemented the protocols on the Quantum network simulator \href{https://netsquid.org/}{\texttt{NetSquid}}
        \item Inspected the performance of distillation protocols and the possibility of entanglement distillation in presence of noisy channels and imperfect initial states (SPAM errors)
        \item Compared the performance of the protocols for near term noisy quantum channels and examined the effects of noise and presence of quantum memory
    \end{itemize}
%    \item \textbf{\textit{Quantum simulations on IBM Quantum Computer}} \hfill \textit{June'19-Aug'19}\\
%    \texttt{Mentor:} \textit{Prof. P.K. Panigrahi, Physicsal Sciences, IISER Kolkata} \hfill
%    \begin{itemize}
%        \item Learned perturbation techniques and examined the theoretical models proposed to understand photoluminescence and energy levels in Quantum Dots
%        \item Employed IBM Quantum Experience and got hands on knowledge of simulations using the five-qubit quantum computer with different noise models
%        \item Modelled and solved simplified lower dimensional versions of the three-dimensional system to gain insight into higher-dimensional phenomena. Employed Qiskit to obtain solutions for more complicated systems
%    \end{itemize}
    \iffalse
\end{itemize}
\color{myBlue}
\vspace{0.2cm}
\noindent \Large\textbf{\textit{Key Projects}} \vspace{0.15cm} \normalsize
\color{gray} \hrule
\color{black}
\begin{itemize}
\fi
%    \item \textbf{\textit{Sliding Window Lempel-Ziv}} \hfill \textit{Oct'19-Nov'19}\\
%    \texttt{Mentor:} \textit{Prof. R.K. Bansal, Dept. of Electrical Engineering, IIT Kanpur} \hfill

    \item \textbf{\textit{Quantum Machine Learning}} \hfill \textit{August'19}\\
    \texttt{Mentor:} \textit{Prof. P.K. Panigrahi, Physicsal Sciences, IISER Kolkata}  \hfill \href{https://link.springer.com/chapter/10.1007/978-981-15-5619-7_8}{\texttt{[LINK]}}
    \begin{itemize}
        \item Studeied Classical ML and the connection to hybrid classical-quantum Machine learning
        \item Analysed Quantum HHL algorithm and its implementation and simulated components of the algorithm on the five-qubit IBM Quantum Computer
        \item Examined classifiers that use classical and quantum machine learning and contrasted them
        \item Co-authored the review paper \textit{Quantum Machine Learning : A Review and Current Status} presented at ICDMAI 2020, New Delhi  \hfill \href{https://www.icdmai.org/}{\texttt{[CONFERENCE]}}
    \end{itemize}
    \item \textbf{\textit{Quantum Key Distribution using BB84 protocol}} \hfill \textit{May'18-July'18}\\
    \texttt{Mentor:} \textit{Prof. Saikat Ghosh, Department of Physics, Indian Institute of Technology, Kanpur} \hfill
    \begin{itemize}
        \item Developed understanding of informtation theory, various coding algorithms and their optimality
        \item Learned about quantum and classical communication protocols such as BB84, SPI, and UART
        \item Set up nultiple sensors such as GPS, accelerometer, and gyroscope and integrated the signal collected from them to deploy a self-aligning network of lasers and detectors for communication
        \item Used an SoC development board (Zybo Z7) with FPGA \& programmed it using Xilinx SDK and Xilinx Vivado to integrate data from sensor modules and run the stepper motor and laser system
        \item Established a classical channel using SPI protocol by low-cost lasers and detectors scavenged from old CD drivers. Designed a circuit on a development board to run the system
        \item Implemented Huffman and variants of Lempel Ziv (LZ77 \& 78) algorithms to encode information
    \end{itemize}
%    \item \textbf{\textit{Non local games \& the set of quantum correlations (Tsirelson's Problem)}} \hfill \textit{Mar'18-Apr'19}\\
%    \texttt{Mentor:} \textit{Prof. Rajat Mittal, Department of Computer Science, IIT, Kanpur} \hfill \href{http://home.iitk.ac.in/~smit/CS682_Presentation.pdf}{\textsc{[SLIDES]}}\\
%    \texttt{Course:} \textit{Quantum Computing (CS682A)} \hfill \href{http://home.iitk.ac.in/~smit/CS682_Smit_Report.pdf}{\textsc{[REPORT]}} \vspace{-0.1cm}
%    \begin{itemize}
%        \item Read about Non-Local games and Quantum Correlations and the implication of Bell's inequality
%        \item Understood quantum correlations using finite and infinite dimensional quantum strategies, in the limit of finite-dimensional strategies and commuting operator strategies, and established the relations between them
%        \item Learnt about the proof of non-closure of set of finite-dimensional quantum correlations from \href{https://arxiv.org/abs/1703.08618}{William Slofstra's paper}
%    \end{itemize}
%    \item \textbf{\textit{Neural network for global and local temperature prediction}} \hfill \textit{Mar'19-Apr-19}\\
%    \texttt{Mentor:} \textit{Prof. Mahendra K. Verma, Department of Physics, IIT Kanpur} \hfill \\ %\href{http://home.iitk.ac.in/~smit/PHY473_Presentation.pdf}{\textsc{[SLIDES]}}\\
%    \texttt{Course:} \textit{Computational Physics (PHY473A)} \hfill %\href{http://home.iitk.ac.in/~smit/PHY473.pdf}{\textsc{[REPORT]}}
%    \begin{itemize}
%        \item Did time-series analysis of temperature of various cities using ARIMA model
%        \item Employed a feed-forward and recurrent neural network using TensorFlow
%        \item Built and implemented a Recurrent Neural Network trained on historic stock prices data and analysed the action of different optimizers, loss function and network depth
%    \end{itemize}
%    \item \textbf{\textit{Bohmian Mechanics and Quantum Information}} \hfill \textit{July'18-Nov'18}\\
%    \texttt{Mentor:} \textit{Prof. Kaushik Bhattacharya, Department of Physics, IIT, Kanpur} \hfill \href{http://home.iitk.ac.in/~smit/UGP_Presentation.pdf}{\textsc{[SLIDES$\mid$}}\href{http://home.iitk.ac.in/~smit/UGP_Final_Report.pdf}{\textsc{REPORT]}} \vspace{-0.1cm}
%    \begin{itemize}
%    \item Read Bohm's original papers on Wave Mechanics and developed an understanding of Bell's inequality and locality and hidden variables (local and non-local)
%    \item Studied about the Aharanov Bohm Effect and Landau levels, created simulations for the system and reproduced the results and showed quantization of energy levels
%    \item Examined the relation between Boltzman's statistical ideas, (typicality and chance) to Bohmian postulates and Bohmian Mechanics
%    \item Investigated the implications and predictions of Bohm's theory in non-relativistic settings. Inspected the proof of non-existence of local hidden variables
%    \end{itemize}

%    \item \textbf{\textit{Emulating particle in finite well using Analog Circuit}} \hfill \textit{July'18-Nov'18}\\
%    \texttt{Course:} \textit{Modern Physics Laboratory (PHY315A)} \hfill \href{http://home.iitk.ac.in/~smit/Final-Presentation.pdf}{\textsc{[POSTER$\mid$}}\href{http://home.iitk.ac.in/~smit/PHY315_Final_Report.pdf}{\textsc{REPORT]}}
%    \begin{itemize}
%        \item Devised a circuit that mapped variables (such as position and wave amplitude) in a \textit{Particle in a Box} problem to currents and voltages at various points of circuit
%        \item Created a feedback mechanism to perform \textit{numerical integration} and provide control parameters to avail initial conditions using an analog circuit
%        \item Visualized the scaled \textit{Wave amplitude} on an oscilloscope for various energies and calculated the scaled eigen energies according to the behavior of the wave.
%    \end{itemize}
    \item \textbf{\textit{Temperature dependence of refractive index of liquids}} \hfill \textit{July-Nov'17}\\
    \texttt{Mentor:} \textit{Prof. Saikat Ghosh, Department of Physics, IIT Kanpur}\hfill \href{https://github.com/smitchaudhary/CV/blob/main/Assets/PHY224_Project_Report.pdf}{\textsc{[REPORT]}}\\
    \texttt{Course:} \textit{Optics (PHY224A)}
    \begin{itemize}
        \item Devised and set up a modified version of \textbf{Michelson's interferometer} with a column of liquid along one arm of the interferometer
        \item Observed the collapsing circular fringes with changing temperature of the liquid column placed along one of the arms of the interferometer
        \item Calculated the change in refractive index with changing temperature using the number of fringes collapsed with each degree Celsius change in temperature
    \end{itemize}

\end{itemize}
%\color{myBlue}
%\noindent \Large\textbf{\textit{Relevant Coursework}} \vspace{0.1cm}
%\hrule \normalsize
%\color{black}
%\begin{table}[!h]
%\begin{tabular}{@{\phantom{x}}l@{\phantom{XX}}l@{\phantom{XX}}l@{\phantom{XX}}l@{\phantom{x}}}
%Quantum Mechanics         & Quantum Computing & Quantum Field Theory        & Quantum information    \\
%Machine Learning & Information Theory & Computational Physics & Linear Algebra\\
%Statistical Mechanics & Modern Physics\textsuperscript{\#}& Optics\textsuperscript{\#}        & Classical %Mechanics\\
%Probability \& Statistics & Quantum Communication & Introductory Electronics\textsuperscript{\#} & General %Relativity \\
%\end{tabular}
%\end{table} \vspace{-0.35cm}
%\hfill \small{\textit{\# - Lab Course}} \phantom{XXXXXXXX} \\
\color{myBlue}
\noindent \Large\textbf{\textit{Technical Skills}} \vspace{0.15cm}
\hrule \normalsize
\color{black} \vspace{0.3cm}
\textbf{Programming:} Python, C/C++, Verilog

\textbf{Utilities:} Qiskit, Pennylane, MATLAB, \LaTeX, Vivado Design Suite, Arduino, Mathematica \\

\color{myBlue}
\noindent \Large\textbf{\textit{Teaching and Co-curricular activiteis}} \vspace{0.15cm}
\hrule \normalsize
\color{black} \vspace{0.3cm}
\begin{itemize}
  \item \textbf{Teaching Assistant:} NB2211 - Electronics Instrumentation, TU Delft \hfill 2020-21 \& 2021-22\vspace{-0.17cm}
  \item \textbf{Volunteer Teacher:} Volunteer teacher for under-priviled students from villages near IIT Kanpur \vspace{-0.17cm}
  \item \textbf{Editor, Vox Populi:} Editor of \href{https://voxiitk.com/}{\textit{Vox Populi}}, the student journalism body of IIT Kanpur
\end{itemize}
\end{document}
